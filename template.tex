%%%%%%%%%%%%%%%%%%%%%%%%%%%%%%%%%%%%%%%%%
% Focus Beamer Presentation
% LaTeX Template
% Version 1.0 (8/8/18)
%
% This template has been downloaded from:
% http://www.LaTeXTemplates.com
%
% Original author:
% Pasquale Africa (https://github.com/elauksap/focus-beamertheme) with modifications by
% Vel (vel@LaTeXTemplates.com)
%
% Template license:
% GNU GPL v3.0 License
%
% Important note:
% The bibliography/references need to be compiled with bibtex.
%
%%%%%%%%%%%%%%%%%%%%%%%%%%%%%%%%%%%%%%%%%

%----------------------------------------------------------------------------------------
%	PACKAGES AND OTHER DOCUMENT CONFIGURATIONS
%----------------------------------------------------------------------------------------

\documentclass{beamer}

\usetheme{focus} % Use the Focus theme supplied with the template
% Add option [numbering=none] to disable the footer progress bar
% Add option [numbering=fullbar] to show the footer progress bar as always full with a slide count

% Uncomment to enable the ice-blue theme
%\definecolor{main}{RGB}{92, 138, 168}
%\definecolor{background}{RGB}{240, 247, 255}

%------------------------------------------------

\usepackage[french]{babel}
\usepackage{booktabs} % Required for better table rules
\usepackage{csquotes}
\usepackage{emoji}
\usepackage{hyperref}
\usepackage{minted}
\usepackage[T1]{fontenc}
\usepackage{svg} % to access the named colour LightGray
\usepackage{xcolor} % to access the named colour LightGray
\definecolor{LightGray}{gray}{0.9}

\hypersetup{
    pdfborderstyle={/S/U/W 1}
}


\AtBeginSection[]
{\begin{frame}
    \frametitle{Sommaire}
    \tableofcontents[currentsection]
\end{frame}
}
%----------------------------------------------------------------------------------------
%	 TITLE SLIDE
%----------------------------------------------------------------------------------------

\title{Refonte graphique des emplois}

\subtitle{Réusiner sereinement}

\author{\href{mailto:mail@franek.fr}{François FREITAG}}

\titlegraphic{\vspace{25pt} \frame{\includegraphics[scale=0.22]{Images/htmx.png}}}

\institute{Les emplois de l'inclusion \\ GIP Inclusion}

\date{7 octobre 2024}

%------------------------------------------------

\begin{document}

%------------------------------------------------

\begin{frame}
	\maketitle % Automatically created using the information in the commands above
\end{frame}

%----------------------------------------------------------------------------------------
%	 SECTION 1
%----------------------------------------------------------------------------------------

\section{Variables non définies templates Django} % Section title slide, unnumbered

\subsection*{Introduction}

\begin{frame}{Plongée dans django ?}
    Mécanisme de vérification des variables au rendu des templates django

    \vspace{1cm}

    \textit{\huge Intéressés ?}

    \vspace{.5cm}
    \begin{itemize}
        \item {\large \textbf{Oui} : Restez}
        \item {\large \textbf{Non} : \emoji{coffee} [30 minutes]}
    \end{itemize}
\end{frame}

\begin{frame}[fragile]
    \frametitle{Gestion des variables non définies}

    Qu'affiche ?
    \begin{minted}[bgcolor=LightGray,fontsize=\scriptsize]{django}
        <div>
          {{ does_not_exist }}
        </div>
    \end{minted}

    \pause

    \begin{minted}[bgcolor=LightGray,fontsize=\scriptsize]{html}
        <div>

        </div>
    \end{minted}
    \pause

    Pourquoi ?
\end{frame}

\begin{frame}{\texttt{string\_if\_invalid}}
    \foreignquote{english}{
        Generally, if a variable doesn't exist, the template system inserts the
        value of the engine's \texttt{string\_if\_invalid} configuration
        option, which is set to '' (the empty \texttt{string)} by default.

        \vspace{0.5cm}

        Filters that are applied to an invalid variable will only be applied if
        \texttt{string\_if\_invalid} is set to '' (the empty string). If
        \texttt{string\_if\_invalid} is set to any other value, variable
        filters will be ignored.

        \vspace{0.5cm}

        This behavior is slightly different for the \texttt{if}, \texttt{for}
        and \texttt{regroup} template tags. If an invalid variable is provided
        to one of these template tags, the variable will be interpreted as
        \texttt{None}. Filters are always applied to invalid variables within
        these template tags.

        \vspace{0.5cm}

        If \texttt{string\_if\_invalid} contains a '\%s', the format marker
        will be replaced with the name of the invalid variable.
    }
\end{frame}

\begin{frame}{\texttt{pytest-django} \texttt{fail-on-template-vars}}
    Le harnais \texttt{pytest-django} fournit une option pour faire échouer un
    test lorsqu'un template utilise une variable non définie:

    \vspace{0.5cm}

    \foreignquote{english}{
        \texttt{fail-on-template-vars}: fail for invalid variables in templates
    }

    \vspace{0.5cm}

    Cet utilitaire est imparfait, et a été amélioré par Xavier.
    \begin{itemize}
        \item \href{https://github.com/pytest-dev/pytest-django/issues/471}{interfère avec les \texttt{OneToOneField}s}
        \item \href{https://github.com/pytest-dev/pytest-django/pull/1130}{ignore les filtres \texttt{|default:}}
    \end{itemize}
\end{frame}

\subsection*{Rendu des templates django}

\begin{frame}[fragile]
    \frametitle{Rendu des templates: moteur}

    \begin{minted}[bgcolor=LightGray,fontsize=\scriptsize]{python}
class Engine:  # django.template.engine
    def from_string(self, template_code):
        return Template(template_code, engine=self)

class Template:
    def __init__(self, template_string, **kwargs):
        ...
        self.source = str(template_string)
        self.nodelist = self.compile_nodelist()

    def compile_nodelist(self):
        # Slighly simplified:
        lexer = Lexer(self.source)
        tokens = lexer.tokenize()
        parser = Parser(
            tokens,
            self.engine.template_libraries,
            self.engine.template_builtins,
            self.origin,
        )
        return parser.parse()
    \end{minted}
\end{frame}

\begin{frame}[fragile]
    \frametitle{Rendu des templates: parser}

    \vspace{-0.3cm}

    \begin{minted}[bgcolor=LightGray,fontsize=\tiny]{python}
class Parser:
    def __init__(self, tokens, **kwargs):
        # Reverse the tokens so delete_first_token(), prepend_token(), and
        # next_token() can operate at the end of the list in constant time.
        self.tokens = list(reversed(tokens))
        ...

    def compile_filter(self):
        return FilterExpression(token, self)

    def parse(self, **kwargs):
        nodelist = NodeList()
        while self.tokens:
            token = self.next_token()
            # Use the raw values here for TokenType.* for a tiny performance boost.
            token_type = token.token_type.value
            if token_type == 0:  # TokenType.TEXT
                self.extend_nodelist(nodelist, TextNode(token.contents), token)
            elif token_type == 1:  # TokenType.VAR
                if not token.contents:
                    raise self.error(token, "Empty variable tag on line %d" % token.lineno)
                try:
                    filter_expression = self.compile_filter(token.contents)
                except TemplateSyntaxError as e:
                    raise self.error(token, e)
                var_node = VariableNode(filter_expression)
                self.extend_nodelist(nodelist, var_node, token)
            elif token_type == 2:  # TokenType.BLOCK
                pass  # We can ignore the rest.
    \end{minted}
\end{frame}

\begin{frame}[fragile]
    \frametitle{\texttt{FilterExpression}}

    \begin{minted}[bgcolor=LightGray,fontsize=\scriptsize]{python}
class FilterExpression:
    """Parse a variable token and its optional filters [...]"""
    # {{ user }}, {{ user | default:None }}
    def resolve(self, context, ignore_failures=False):
        ...
        try:
            obj = self.var.resolve(context)
        except VariableDoesNotExist:
            if ignore_failures:
                # FirstOfNode, ForNode, IfChangedNode, IfNode, RegroupNode.
                obj = None
            else:
                string_if_invalid = context.template.engine.string_if_invalid
                if string_if_invalid:
                    if "%s" in string_if_invalid:
                        return string_if_invalid % self.var
                    else:
                        return string_if_invalid
                else:
                    obj = string_if_invalid  # 99.9999... % of templates.
        ... # Apply template filters sequentially.
    \end{minted}
\end{frame}

\subsection*{Présentation fail-on-template-vars}

\begin{frame}[fragile]
    \frametitle{Fonctionnement \texttt{fail-on-template-vars}}

    La doc de django stipule:
    \foreignquote{english}{
        If \texttt{string\_if\_invalid} contains a \texttt{'\%s'}, the format
        marker will be replaced with the name of the invalid variable.
    }

    \vspace{1cm}

    \href{https://github.com/pytest-dev/pytest-django/blob/1ffc3239387bcfadbd8aecfdfacf3d6b52b947f8/pytest_django/plugin.py\#L661}{Idée diabolique} :
    \begin{itemize}
        \item créer un objet personnalisé injecté en tant que \texttt{string\_if\_invalid}
        \item surcharger \texttt{\_\_contains\_\_} pour répondre \texttt{True} pour passer :
\begin{minted}[bgcolor=LightGray,fontsize=\scriptsize]{python}
if "%s" in string_if_invalid:
\end{minted}

        \item surcharger \texttt{\_\_mod\_\_} pour \texttt{self.fail()} le test :
\begin{minted}[bgcolor=LightGray,fontsize=\scriptsize]{python}
    return string_if_invalid % self.var
\end{minted}
    \end{itemize}
\end{frame}

\begin{frame}[fragile]
    \frametitle{Les deux problèmes}
    \begin{itemize}
        \item \href{https://github.com/pytest-dev/pytest-django/issues/471}{Les tests échouent si une variable n'existe pas}, même s'il y a un filtre \texttt{|default:}
\begin{minted}[bgcolor=LightGray,fontsize=\scriptsize]{django}
<p>{{ phone|default:"Non renseigné" }}</p>
\end{minted}
        \item \href{https://github.com/pytest-dev/pytest-django/pull/1130}{Les \texttt{OneToOneField} dans les blocks \texttt{if} sont toujours évalués \texttt{True}}
\begin{minted}[bgcolor=LightGray,fontsize=\scriptsize]{django}

    Manager : {{ user.supervisor }}

\end{minted}
    \end{itemize}
\end{frame}

\begin{frame}[fragile]
    \frametitle{\texttt{fail-on-template-vars}: \texttt{Variable}}
    \vspace{-0.3cm}

    \begin{minted}[bgcolor=LightGray,fontsize=\tiny]{python}
class Variable:
    # 
    def _resolve_lookup(self, context):
        """Perform resolution of a real variable (i.e. not a literal) against the given context."""
        current = context
        try:  # catch-all for silent variable failures
            for bit in self.lookups:  # self.lookups = ["user", "supervisor"]
                try:  # dictionary lookup
                    current = current[bit]
                except (TypeError, AttributeError, KeyError, ValueError, IndexError):
                    try:  # attribute lookup
                        ...
                        current = getattr(current, bit)
                    except (TypeError, AttributeError):
                        ...
                        try:  # list-index lookup
                            current = current[int(bit)]
                        except (IndexError, ValueError, KeyError, TypeError):
                            raise VariableDoesNotExist(
                                "Failed lookup for key [%s] in %r", (bit, current)
                            )
                # ... handle callables
        except Exception as e:
            template_name = getattr(context, "template_name", None) or "unknown"
            if getattr(e, "silent_variable_failure", False): # ObjectDoesNotExist and subclasses.
                current = context.template.engine.string_if_invalid  # Truthy when patched!!
    \end{minted}
\end{frame}

\begin{frame}[fragile]
    \frametitle{\texttt{FilterExpression}}

    \begin{minted}[bgcolor=LightGray,fontsize=\scriptsize]{python}
class FilterExpression:
    """Parse a variable token and its optional filters [...]"""
    # {{ user }}, {{ user | default:None }}
    def resolve(self, context, ignore_failures=False):
        ...
        try:
            obj = self.var.resolve(context)
        except VariableDoesNotExist:
            if ignore_failures:
                # FirstOfNode, ForNode, IfChangedNode, IfNode, RegroupNode.
                obj = None
            else:
                string_if_invalid = context.template.engine.string_if_invalid
                if string_if_invalid:
                    if "%s" in string_if_invalid:
                        return string_if_invalid % self.var
                    else:
                        return string_if_invalid
                else:
                    obj = string_if_invalid  # 99.9999... % of templates.
        ...
    \end{minted}
\end{frame}

\subsection*{Solution des deux problèm̀es}

\begin{frame}[fragile]
    \frametitle{\href{https://github.com/gip-inclusion/les-emplois/blob/c34a60b8dfd24418cbd22b6ad50101f377c1adb7/tests/conftest.py\#L289-L317}{Patch 1}: \texttt{|default} and \texttt{OneToOneField}}

    \begin{minted}[bgcolor=LightGray,fontsize=\tiny]{python}
@pytest.fixture(autouse=True, scope="session")
def _fail_for_invalid_template_variable_improved(_fail_for_invalid_template_variable):
    # Edge cases stuff omitted to fit on the slide.
    from django.conf import settings as dj_settings

    invalid_var_exception = dj_settings.TEMPLATES[0]["OPTIONS"]["string_if_invalid"]

    # Make InvalidVarException falsy to keep the behavior consistent for OneToOneField
    invalid_var_exception.__class__.__bool__ = lambda self: False

    # but adapt Django's template code to behave as if it was truthy in resolve
    # (except when the default filter is used)
    patchy.patch(
        base_template.FilterExpression.resolve,
        """\
        @@ -7,7 +7,8 @@
                         obj = None
                     else:
                         string_if_invalid = context.template.engine.string_if_invalid
        -                if string_if_invalid:
        +                from django.template.defaultfilters import default as default_filter
        +                if default_filter not in {func for func, _args in self.filters}:
                             if "%s" in string_if_invalid:
                                 return string_if_invalid % self.var
                             else:
        """,
    )
    \end{minted}
\end{frame}

\begin{frame}[fragile]
    \frametitle{FAIL\_INVALID\_TEMPLATE\_VARS}
    \begin{minted}[bgcolor=LightGray,fontsize=\tiny]{toml}
[tool.pytest.ini_options]
FAIL_INVALID_TEMPLATE_VARS = true
    \end{minted}

    \pause

    Des tests échouent ? Établir une baseline, avec :

    \begin{minted}[bgcolor=LightGray,fontsize=\tiny]{python}
@pytest.mark.ignore_template_errors
def test_expected_failure():
    pass
    \end{minted}
\end{frame}

\subsection*{Résolution des variables dans les \texttt{if}}

\begin{frame}[fragile]
    \frametitle{Quizz}
    \begin{minted}[bgcolor=LightGray]{django}

    <p>More content</p>

    \end{minted}
    \pause
    \foreignquote{english}{
        This behavior is slightly different for the \texttt{if}, \texttt{for}
        and \texttt{regroup} template tags. If an invalid variable is provided
        to one of these template tags, the variable will be interpreted as
        \texttt{None}. Filters are always applied to invalid variables within
        these template tags.
    }
\end{frame}

\begin{frame}[fragile]
    \frametitle{\texttt{fail-on-template-vars}: \texttt{FilterExpression}}

    \begin{minted}[bgcolor=LightGray,fontsize=\tiny]{python}
@register.tag("if")
def do_if(parser, token):
    # 
    bits = token.split_contents()[1:]
    condition = TemplateIfParser(parser, bits).parse()
    nodelist = parser.parse(("elif", "else", "endif"))
    conditions_nodelists = [(condition, nodelist)]
    token = parser.next_token()
    #  (repeatable)
    ...
    #  (optional)
    ...
    # 
    ...
    return IfNode(conditions_nodelists)

class TemplateIfParser(IfParser):
    def create_var(self, value):
        return TemplateLiteral(self.template_parser.compile_filter(value), value)

class TemplateLiteral(Literal):
    def __init__(self, value, text):
        self.value = value
        self.text = text  # for better error messages

    def display(self):
        return self.text

    def eval(self, context):
        return self.value.resolve(context, ignore_failures=True)
    \end{minted}
\end{frame}


\begin{frame}[fragile]
    \frametitle{\texttt{FilterExpression}}

    \begin{minted}[bgcolor=LightGray,fontsize=\tiny]{python}
class FilterExpression:
    """Parse a variable token and its optional filters [...]"""
    # {{ user }}, {{ user | default:None }}
    def resolve(self, context, ignore_failures=False):
        ...
        try:
            obj = self.var.resolve(context)
        except VariableDoesNotExist:
            if ignore_failures:
                # FirstOfNode, ForNode, IfChangedNode, IfNode, RegroupNode.
                obj = None
            else:
                string_if_invalid = context.template.engine.string_if_invalid
                if string_if_invalid:
                    if "%s" in string_if_invalid:
                        return string_if_invalid % self.var
                    else:
                        return string_if_invalid
                else:
                    obj = string_if_invalid  # 99.9999... % of templates.
        ...
    \end{minted}
\end{frame}

\subsection*{Solution résolution variables dans les \texttt{if}}

\begin{frame}[fragile]
    \frametitle{\href{https://github.com/gip-inclusion/les-emplois/blob/2ccc5db10f07c6049c66a96f5fe57ad6715bdc91/tests/conftest.py\#L320-L350}{Patch 2}: Détecter plus d'erreur}

    \begin{minted}[bgcolor=LightGray,fontsize=\tiny]{python}
@pytest.fixture(autouse=True, scope="function")
def unknown_variable_template_error(monkeypatch, request):
    marker = request.keywords.get("ignore_unknown_variable_template_error", None)
    BASE_IGNORE_LIST = {"debug", "user"}
    strict = True
    if marker is None:
        ignore_list = BASE_IGNORE_LIST
    elif marker.args:
        ignore_list = BASE_IGNORE_LIST | set(marker.args)
    else:
        # Marker without list
        strict = False

    if strict:
        origin_resolve = base_template.FilterExpression.resolve
        # FirstOfNode, ForNode, IfChangedNode, IfNode, RegroupNode all force ignore_failures=True.
        def stricter_resolve(self, context, ignore_failures=False):
            if (
                self.is_var
                and self.var.lookups is not None
                and self.var.lookups[0] not in context
                and self.var.lookups[0] not in ignore_list
            ):
                ignore_failures = False
            return origin_resolve(self, context, ignore_failures)

        monkeypatch.setattr(base_template.FilterExpression, "resolve", stricter_resolve)
    \end{minted}
\end{frame}

\section{Modernisation avec HTMX}

\subsection*{Avant}

\begin{frame}[fragile]
    \frametitle{Le vénérable auto-submit}

    Recharger la page à chaque changement.

    \href{http://localhost:8001/}{Démo}
    \begin{minted}[bgcolor=LightGray,fontsize=\scriptsize]{javascript}
function submitFiltersForm() {
  $("#js-job-applications-filters-form").submit();
}
$("#js-job-applications-filters-form :input").change(submitFiltersForm);
$("duet-date-picker").on("duetChange", submitFiltersForm);
    \end{minted}
\end{frame}

\subsection*{Présentation HTMX}

\begin{frame}{HTMX}
    Librairie JavaScript
    \begin{itemize}
        \item Fonctionnement typique:
            \begin{enumerate}
                \item Définition des déclencheurs (\texttt{hx-trigger="click"})
                \item Paramétrage du Fetch (url, méthode HTTP, headers, body)
                \item Le serveur génère la réponse (généralement un fragment \texttt{HTML})
                \item Substitution d'une partie du DOM par la réponse
            \end{enumerate}
        \item Décrit par des attributs sur les éléments \texttt{HTML} (\texttt{hx-*})
    \end{itemize}

    \vspace{0.5cm}

    \href{https://htmx.org/examples/click-to-edit/}{\textbf{Démo \texttt{HTMX}}}
\end{frame}

\subsection*{Intégration dans Django}

\begin{frame}{Recharger un fragment}
    \href{https://demo.emplois.inclusion.beta.gouv.fr/}{Démo filtres candidatures}
\end{frame}

\begin{frame}[fragile]
    \frametitle{Template de la page}

    \texttt{apply/job\_application\_list.html}:

    \begin{minted}[bgcolor=LightGray,fontsize=\scriptsize]{django}



    <div class="s-section__row row">
        <div class="col-12">
            <h2 id="job-app-title">
                {{ job_apps|length }} résultat{{ job_apps|pluralize }}
            </h2>
        </div>
        <div class="col-12">
            <form hx-get="/apply/siae/list/"
                  hx-trigger="change"
                  hx-target="#job-app-results">
                {{ form.eligibility }}
            </form>
            
        </div>
    </div>

    \end{minted}
\end{frame}

\begin{frame}[fragile]
    \frametitle{Fragment pour HTMX}

    \texttt{apply/includes/job\_app\_results.html}:
    \begin{minted}[bgcolor=LightGray,fontsize=\scriptsize]{django}
<ul id="job-app-results">
    
        <li>Candidature de {{ job_app.job_seeker.full_name }}</li>
    
</ul>
    \end{minted}
\end{frame}

\begin{frame}[fragile]
    \frametitle{La vue django}

    \begin{minted}[bgcolor=LightGray,fontsize=\scriptsize]{python}
def job_applications(request, *args, , **kwargs):
    # Access controls.
    context = {
        "job_apps": JobApplication.objects.filter(
            to_company=request.user.current_organization,
        )
    }
    template_name = (
        "apply/includes/job_app_results.html"
        if request.htmx
        else "apply/job_application_list.html"
    )
    return render(request, template_name, context)
    \end{minted}
\end{frame}

\subsection*{Out-of-band swap}

\begin{frame}{Démo out-of-band swap}
    \href{https://demo.emplois.inclusion.beta.gouv.fr/}{Démo actions préalables à l'embauche GEIQ}
\end{frame}

\begin{frame}[fragile]
    \frametitle{Template de la page (sans OOB)}

    \texttt{apply/job\_application\_list.html}:

    \begin{minted}[bgcolor=LightGray,fontsize=\scriptsize]{django}



    <div class="s-section__row row">
        <div class="col-12">
            <h2 id="job-app-title">
                {{ job_apps|length }} résultat{{ job_apps|pluralize }}
            </h2>
        </div>
        <div class="col-12">
            <form hx-get="/apply/siae/list/"
                  hx-trigger="change"
                  hx-target="#job-app-results">
                {{ form.eligibility }}
            </form>
            
        </div>
    </div>

    \end{minted}
\end{frame}

\begin{frame}[fragile]
    \frametitle{Template de la page (pour OOB)}

    \texttt{apply/job\_application\_list.html}:

    \begin{minted}[bgcolor=LightGray,fontsize=\scriptsize]{django}



    <div class="s-section__row row">
        <div class="col-12">
            
        </div>
        <div class="col-12">
            <form hx-get="/apply/siae/list/"
                  hx-trigger="change"
                  hx-target="#job-app-results">
                {{ form.eligibility }}
            </form>
            
        </div>
    </div>

    \end{minted}
\end{frame}

\begin{frame}[fragile]
    \frametitle{Fragments pour HTMX}
    \texttt{apply/includes/job\_app\_title.html}
    \begin{minted}[bgcolor=LightGray,fontsize=\scriptsize]{django}
<h2 id="job-app-title" hx-swap-oob="true">
    {{ job_apps|length }} résultat{{ job_apps|pluralize }}
</h2>
    \end{minted}
    \texttt{apply/includes/job\_app\_results.html}:
    \begin{minted}[bgcolor=LightGray,fontsize=\scriptsize]{django}
<ul id="job-app-results">
    
        <li>Candidature de {{ job_app.job_seeker.full_name }}</li>
    
</ul>

    

    \end{minted}
\end{frame}

\begin{frame}{Danger des out-of-band}
    \begin{itemize}
        \item Il est facile d'oublier de mettre à jour un fragment de la page
    \end{itemize}
\end{frame}

\section{Utilitaire de test pour HTMX}

\begin{frame}[fragile]
    \frametitle{\href{https://github.com/gip-inclusion/les-emplois/blob/c05aac7f05d44c0a702076f9694bdf97ab7bfe8c/tests/utils/htmx/tests.py\#L27}{\texttt{update\_page\_with\_htmx}} \hfill{} 1/2}

    \begin{minted}[bgcolor=LightGray,fontsize=\tiny]{python}
def test_update_page_with_htmx():
    simulated_page = parse_response_to_soup(FakeResponse(b"""
          <html><body><div id="main">
                <div id="div-to-swap">
                  <form hx-post="/somewhere?with_query_string=" hx-swap="outerHTML"
                        hx-target="#div-to-swap">
                    <input name="some_input"/>
                  </form>
                </div>
                <div id="other-div-for-oob-test"></div>
          </div></body></html> """))
    htmx_response = FakeResponse(b"""
        <div id="swapped-div">No more form !</div>
        <a href="#">Here is a link instead</a>
        <div id="other-div-for-oob-test" hx-swap-oob="true">OOB Swap successful</div>""",
        request={"PATH_INFO": "/somewhere", "REQUEST_METHOD": "POST", "QUERY_STRING": "with_query_string="},
    )
    update_page_with_htmx(simulated_page, "#div-to-swap > form", htmx_response)
    assertSoupEqual(
        simulated_page,
        BeautifulSoup(b"""
          <html><body><div id="main">
                <div id="swapped-div">No more form !</div>
                <a href="#">Here is a link instead</a>
                <div id="other-div-for-oob-test">OOB Swap successful</div>
          </div></body></html>""", "html5lib")
    )
    \end{minted}
\end{frame}

\begin{frame}[fragile]
    \frametitle{\href{https://github.com/gip-inclusion/les-emplois/blob/9d800e6b3c8a814f17303157966830a0411db8e5/tests/utils/htmx/test.py\#L64}{\texttt{update\_page\_with\_htmx}} \hfill{} 2/2}
    \vspace{-0.3cm}
    \begin{minted}[bgcolor=LightGray,fontsize=\scriptsize]{python}
def test_htmx_with_oob(self):
    self.client.force_login(self.user)
    response = self.client.get(self.URL, {"status": "NEW"})
    simulated_page = parse_response_to_soup(response)

    [new_status] = simulated_page.find_all(
        "input", attrs={"name": "status", "value": "NEW"},
    )
    del new_status["checked"]
    [ready_status] = simulated_page.find_all(
        "input", attrs={"name": "status", "value": "READY"},
    )
    ready_status["checked"] = ""
    response = self.client.get(
        self.URL, {"status": "READY"}, headers={"HX-Request": "true"}
    )
    update_page_with_htmx(
        simulated_page, f"form[hx-get='{self.URL}']", response
    )
    response = self.client.get(self.URL, {"status": "READY"})
    fresh_page = parse_response_to_soup(response)
    assertSoupEqual(simulated_page, fresh_page)
    \end{minted}
\end{frame}

\begin{frame}[fragile]
    \frametitle{\href{https://github.com/gip-inclusion/les-emplois/blob/9d800e6b3c8a814f17303157966830a0411db8e5/tests/utils/htmx/test.py\#L64}{\texttt{update\_page\_with\_htmx}} \hfill 1/2}
    \begin{minted}[bgcolor=LightGray,fontsize=\tiny]{python}
def update_page_with_htmx(page, select_htmx_element, htmx_response):
    [htmx_element] = page.select(select_htmx_element)
    request_method = htmx_response.request["REQUEST_METHOD"]
    if request_method not in ("GET", "POST", "PUT", "DELETE", "PATCH"):
        raise ValueError(f"Unsupported method {request_method}")
    attribute = f"hx-{htmx_response.request['REQUEST_METHOD'].lower()}"
    if attribute not in htmx_element.attrs:
        raise ValueError(f"No {attribute} attribute on provided HTMX element")
    url = htmx_element[attribute]
    if url:
        # If url is "", it means that HTMX will have targeted the current URL
        # https://github.com/bigskysoftware/htmx/blob/v1.8.6/src/htmx.js#L2799-L2802
        # Let's not assert anything in that case, since we currently don't have that info in our test
        parsed_url = urlparse(url)
        assert htmx_response.request["PATH_INFO"] == parsed_url.path
    # We only support HTMX responses that do not try to swap the whole HTML body
    parsed_response = parse_response_to_soup(htmx_response, no_html_body=True)
    \end{minted}
\end{frame}

\begin{frame}[fragile]
    \frametitle{\href{https://github.com/gip-inclusion/les-emplois/blob/9d800e6b3c8a814f17303157966830a0411db8e5/tests/utils/htmx/test.py\#L64}{\texttt{update\_page\_with\_htmx}} \hfill 2/2}
    \begin{minted}[bgcolor=LightGray,fontsize=\tiny]{python}
    out_of_band_swaps = [element.extract() for element in parsed_response.select("[hx-swap-oob]")]
    for out_of_band_swap in out_of_band_swaps:
        oob_swap = out_of_band_swap["hx-swap-oob"]
        target_selector = None
        if oob_swap == "true":
            mode = "outerHTML"
        elif "," in oob_swap:
            mode, target_selector = oob_swap.split(",", maxsplit=1)
        else:
            mode = oob_swap
        del out_of_band_swap["hx-swap-oob"]
        if not target_selector:
            assert out_of_band_swap["id"], out_of_band_swap
            target_selector = f"#{out_of_band_swap['id']}"
        targets = page.select(target_selector)
        for target in targets:
            _handle_swap(page, target=target, new_elements=[out_of_band_swap], mode=mode)
    _handle_swap(
        page,
        target=_get_hx_attribute(htmx_element, "hx-target"),
        new_elements=parsed_response.contents,
        mode=_get_hx_attribute(htmx_element, "hx-swap", default="innerHTML"),
    )
    \end{minted}
\end{frame}

\begin{frame}[fragile]
    \frametitle{\href{https://github.com/gip-inclusion/les-emplois/blob/9d800e6b3c8a814f17303157966830a0411db8e5/tests/utils/htmx/test.py\#L52}{\texttt{\_get\_hx\_attribute}}}
    \begin{minted}[bgcolor=LightGray,fontsize=\tiny]{python}
def _get_hx_attribute(element, attribute, default=None):
    while (value := element.attrs.get(attribute)) is None:
        element = element.parent
        if element is None:
            if default is not None:
                return default
            raise ValueError(f"Attribute {attribute} not found on element or its parents")
    if attribute == "hx-target" and value == "this":
        return element
    return value
    \end{minted}
\end{frame}

\begin{frame}[fragile]
    \frametitle{\href{https://github.com/gip-inclusion/les-emplois/blob/9d800e6b3c8a814f17303157966830a0411db8e5/tests/utils/htmx/test.py\#L37}{\texttt{\_handle\_swap}}}
    \begin{minted}[bgcolor=LightGray,fontsize=\tiny]{python}
def _handle_swap(page, *, target, new_elements, mode):
    if mode == "outerHTML":
        target_element = page.select_one(target) if isinstance(target, str) else target
        if not new_elements:
            # Empty response: remove the target completely
            target_element.decompose()
            return
        [first_element, *rest] = new_elements
        for rest_elt in reversed(rest):
            target_element.insert_after(rest_elt)
        target_element.replace_with(first_element)
        return
    raise NotImplementedError("Other kinds of swap not implemented, please do")
    \end{minted}
\end{frame}

\begin{frame}[focus]
    Merci de votre attention
    \\
    \vspace{20pt}
    \rule{\textwidth}{1pt}
    \\
    \vspace{30pt}
    Avez-vous des questions ?
\end{frame}

\end{document}
